\documentclass[a4paper,11pt]{article}
\usepackage[utf8]{inputenc}
\usepackage{times}
\usepackage[frenchb]{babel}
\usepackage[T1]{fontenc}
\usepackage{lmodern}
\usepackage{graphicx}

\sloppy

\title{Le problème du voyageur de commerce : Implémentation d'un algorithme de résolution}
\author{Guéneau Armaël \and Gledel Valentin}
\date{\today}

\begin{document}
\maketitle

\tableofcontents

\newpage

\addcontentsline{toc}{section}{Introduction}
\section*{Introduction}

Le « Voyageur du commerce » est un problème algorithmique classique, pour lequel de nombreuses solutions approchées existent , bien que la résolution exacte soit difficile. Celui-ci peut se modéliser par un voyageur de commerce possédant une liste de villes à visiter pour vendre ses produits, et qui cherche à minimiser le temps qu’il passe sur la route. Connaissant la liste des villes et les distances entres elles, on veut alors trouver l’ordre dans lequel les parcourir pour que la distance totale parcourue soit minimale. Ce problème est NP-complet, il nous est donc impossible de trouver un algorithme trouvant une solution optimale et fonctionnant en un temps raisonnable.
Dans le cadre du projet~1 en C, nous avons alors implémenté en binôme un algorithme de résolution approchée de ce problème, ainsi qu’une interface utilisateur permettant de sélectionner aisément les villes à parcourir par le voyageur dans une base de donnée.
L’algorithme choisi repose sur le parcours d'un arbre couvrant minimal pour le graphe des villes, construit à l'aide de l'algorithme de Kruskal. Il nous a fallu ensuite implémenter une base de données stockant des villes à parcourir par le voyageur et une interface utilisateur.
Nous allons tout d’abord présenter les algorithmes que nous avons cherché à implémenter tout au long du projet : ceux-ci reposant sur des structures sous-jacentes, celà fera apparaître les contraintes sur celles-ci, ce qui sera attendu a priori. Alors, nous pourrons détailler leur implémentation, sous forme de bibliothèques indépendantes, prêtes à être utilisées ultérieurement. Ceci étant fait, et des bibliothèques propres à notre disposition, nous pourrons expliciter notre implémentation des algorithmes utilisant ces structures. Finalement, nous parlerons des problématiques liées au développement en langage de programmation C, les bibliothèques et les outils utilisés.

\section{Algorithmes}

\subsection{TSP}

\subsection{Algorithmes de Kruskal}

\subsection{Parcours de tries}

\section{Les structures de donées}

Une fois les algorithmes choisis (TSP nous était imposé et Kruskal fortement imposé), il nous fallait choisir les structuresde données adaptées. La plupart des structures ont été conçues sous forme de bibliothèques de sorte à pouvoir éventuellement reservir pour d'autes projets. Cependant, pour l'algorithme de kruskal nous avons développé des structures spécifiques.

\subsection{Les vector}

\subsection{Le graph}

\subsection{Arbres non-enracinée}

\subsection{Structures utilisées dans l'algorithme de Kruskal}

\subsubsection{Listes d'arêtes}

\subsubsection{Forêts}

\subsection{Les tries}

\section{Implémentation des algorithmes}

\subsection{TSP}

\subsection{Algorithmes de Kruskal}

\subsection{Parcours de tries}

\section{Problématiques pratiques de développement : bibliothèques et outils utilisés}


\addcontentsline{toc}{section}{Conclusion}
\section*{Conclusion}


\end{document}
